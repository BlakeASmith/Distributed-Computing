%% Generated by Sphinx.
\def\sphinxdocclass{report}
\documentclass[letterpaper,10pt,english,openany,oneside]{sphinxmanual}
\ifdefined\pdfpxdimen
   \let\sphinxpxdimen\pdfpxdimen\else\newdimen\sphinxpxdimen
\fi \sphinxpxdimen=.75bp\relax

\PassOptionsToPackage{warn}{textcomp}
\usepackage[utf8]{inputenc}
\ifdefined\DeclareUnicodeCharacter
% support both utf8 and utf8x syntaxes
  \ifdefined\DeclareUnicodeCharacterAsOptional
    \def\sphinxDUC#1{\DeclareUnicodeCharacter{"#1}}
  \else
    \let\sphinxDUC\DeclareUnicodeCharacter
  \fi
  \sphinxDUC{00A0}{\nobreakspace}
  \sphinxDUC{2500}{\sphinxunichar{2500}}
  \sphinxDUC{2502}{\sphinxunichar{2502}}
  \sphinxDUC{2514}{\sphinxunichar{2514}}
  \sphinxDUC{251C}{\sphinxunichar{251C}}
  \sphinxDUC{2572}{\textbackslash}
\fi
\usepackage{cmap}
\usepackage[T1]{fontenc}
\usepackage{amsmath,amssymb,amstext}
\usepackage{babel}



\usepackage{times}
\expandafter\ifx\csname T@LGR\endcsname\relax
\else
% LGR was declared as font encoding
  \substitutefont{LGR}{\rmdefault}{cmr}
  \substitutefont{LGR}{\sfdefault}{cmss}
  \substitutefont{LGR}{\ttdefault}{cmtt}
\fi
\expandafter\ifx\csname T@X2\endcsname\relax
  \expandafter\ifx\csname T@T2A\endcsname\relax
  \else
  % T2A was declared as font encoding
    \substitutefont{T2A}{\rmdefault}{cmr}
    \substitutefont{T2A}{\sfdefault}{cmss}
    \substitutefont{T2A}{\ttdefault}{cmtt}
  \fi
\else
% X2 was declared as font encoding
  \substitutefont{X2}{\rmdefault}{cmr}
  \substitutefont{X2}{\sfdefault}{cmss}
  \substitutefont{X2}{\ttdefault}{cmtt}
\fi


\usepackage[Bjarne]{fncychap}
\usepackage{sphinx}

\fvset{fontsize=\small}
\usepackage{geometry}


% Include hyperref last.
\usepackage{hyperref}
% Fix anchor placement for figures with captions.
\usepackage{hypcap}% it must be loaded after hyperref.
% Set up styles of URL: it should be placed after hyperref.
\urlstyle{same}

\usepackage{sphinxmessages}




\title{Lab 3}
\date{May 30, 2020}
\release{}
\author{Blake Smith}
\newcommand{\sphinxlogo}{\vbox{}}
\renewcommand{\releasename}{}
\makeindex
\begin{document}

\pagestyle{empty}
\sphinxmaketitle
\pagestyle{plain}
\sphinxtableofcontents
\pagestyle{normal}
\phantomsection\label{\detokenize{index::doc}}



\chapter{Sequential Map Reduce, A Baseline}
\label{\detokenize{lab3report:sequential-map-reduce-a-baseline}}
Before proceeding with testing my implementation from Lab 2
I will first gather some metrics from
the sequential implementation given in \sphinxtitleref{mrseqential.go}.
This baseline will be used to determine at what point the
multi\sphinxhyphen{}process implementation becomes worthwhile and what
the trade offs are in regards to space \& time complexity.


\section{The Phases of Map Reduce (In the Sequential Case)}
\label{\detokenize{lab3report:the-phases-of-map-reduce-in-the-sequential-case}}
In order to estimate the performance of a sequential map\sphinxhyphen{}reduce
I will be collecting the following information at each step
in the execution.
\begin{enumerate}
\sphinxsetlistlabels{\arabic}{enumi}{enumii}{}{.}%
\item {} 
Read input files and pass into the map function, producing
a collection of intermediate values.
\begin{itemize}
\item {} 
Time taken to read in the input files and produce
the intermediate collection.

\item {} 
Space required to store the intermediate values

\item {} 
Time taken to sort the intermediate values

\end{itemize}

\item {} 
Group the intermediate values by key, producing a list of
values for every key.
\begin{itemize}
\item {} 
Time taken to group the values by key

\item {} 
Amount of memory used in that process

\end{itemize}

\item {} 
Run Reduce on each key and create a single output file
\begin{itemize}
\item {} 
Time taken to complete all reduce jobs and produce
the full output

\end{itemize}

\end{enumerate}


\section{Modifications to \sphinxtitleref{mrsequential.go}}
\label{\detokenize{lab3report:modifications-to-mrsequential-go}}


\renewcommand{\indexname}{Index}
\printindex
\end{document}